\documentclass[a4paper,12pt]{article}

\usepackage[]{hyperref}
%\usepackage{cleveref}

\usepackage[sorting=none,backend=bibtex]{biblatex}
\bibliography{akb29_report}

\usepackage{times}
\usepackage{graphicx,epsfig}
\usepackage[leftcaption]{sidecap}
\usepackage{subfigure} % figures can have sub chunks
\usepackage{geometry} % this maxes page usage, making the below unnecessary
\textwidth = 6.75in
\oddsidemargin = -0.25in
\textheight = 10in
\topmargin = -0.5in
\usepackage{fancyhdr}
\pagestyle{fancy}
\lhead{{\it Alex Birch (with Chris Clarke)}}
\chead{Wall Following LEGO Robots}
\rhead{Coursework 1}
\lfoot{}
\cfoot{\thepage}
\rfoot{}

\newcommand{\goodgap}{%
 \hspace{\subfigtopskip}%
 \hspace{\subfigbottomskip}}

\newcommand{\citeauthoryear}[1]{%
 \citeauthor{#1}%
 ~(\citeyear{#1})}

\title{Coursework 1:  Wall Following with a LEGO Robot}
\author{Alex Birch (with partner Chris Clarke)}


\begin{document}
\maketitle

\section{Introduction}
We adopted a reactive approach to wall-following, for two reasons:
\paragraph{It is difficult to be sure where we are in the world} We could not reliably follow the robot's movement through the world, as its proprioception (the ability for the robot to keep track of its body parts)\cite{lee2002proprioception} was weak; it could not say how far it had moved, even by counting the turns of its wheels. Skidding could make it go less distance than it believed, or its tyres could climb a vertical wall and believe it had travelled horizontally.
\paragraph{It is dangerous to assume what the world looks like} Truly testing intelligence requires robots to exist in the real world. But we cannot produce valid intelligence by planning actions based on a representation of the world, since no representation could wholly represent the world\cite{brooks1991intelligence}. Reacting is a more robust way to cope with a dynamic world.
\\
\\
For collision response, we recruited a mixture of avoidance and recovery. Our robot was designed to follow walls on the left only. It was largely blind on the right-hand-side, but invested a higher resolution in the side it was aiming to follow.
\paragraph{Avoidance} An ultrasound sensor watched the left-hand-normal of the chassis. This was used to track distance to the wall it was following. The purpose was to ensure that the robot ran parallel to the wall, which had three applications: reducing the number of collisions to recover from (so that it could spend more time following the wall), maintaining known angle to wall (to compensate for lack of proprioceptory feedback from motors), and detecting when the wall has ended on the left (to initiate maneuver for continuing to follow it).
\paragraph{Recovery} A bump sensor array was mounted in front of the robot. It was biased to covering the left-hand-side, as that was the side we were following (and thus most anticipating collision with). Collision with this would inform that a wall had been reached, and that it could be followed. The choice of recovery over avoidance for frontal collisions was because a bumper can physically resist further movement into the obstacle, simplifying the recovery. This is an example of how agent morphology generates its own sensory stimulation\cite{pfeifer2005morphological}, and how valid intelligence requires a body\cite{brooks1991intelligence,pfeifer2005morphological}.
%A bump sensor was mounted far in front of the robot. Its purpose was to detect frontal collisions early enough to prevent the wheels gaining purchase on the wall. Since this distance limit was tangible, collision with the sensor would resist further movement in the same direction. 
%This was to support the position-agnostic assumptions we made. 

A subsumption architecture\cite{brooks1991intelligence} complemented our reactive approach. It allowed lower layers (such as `recover from collision') to inhibit immediate goals (such as `follow wall'). This was critical for suppressing actions that were generating inappropriate behaviour. Sensors were all running, all the time, and created the impulse for action. In this way, our action was coupled to perception; our `modules' were behaviours.

Ultrasound sensing was known to be noisy, so we did not initiate maneuvers based on single samples; a percentage of positive readings over some timeframe was required to produce a strong enough impulse to invoke action.

Overall the goal was to 
\section{Approach}
The robot had three motors; two flanking on the rear, and one frontally in the middle. It was hoped that the extra motor could contribute to the robot's speed, especially during steering, where one flanking motor has to compromise its speed output to produce a turn.

The robot was told to begin by walking forward until a collision occurred. The first object it touched was considered `the wall', and its goal was to keep that wall on its left, and follow parallel to it. The `following' consisted of constant range comparisons between current distance and historical distance. If distance from wall was increasing over time, the robot recognised that it was diverging from the wall, so biased its flanking motors to steer toward or away from the wall, maintaining a parallel angle. This fine correction of steering was called `pronation'.

In the case of a right-turn in the wall, a frontal collision was expected to be sensed by the bump sensor, and a `face right' three-point-turn primitive was used to face its right-hand normal. In the case of a left-turn in the wall, a continuous reading of `distance\_max' from the left-hand sensor was expected, and a left three-point-turn primitive was used to turn into the space. Overshooting or undershooting the turn did not matter (whether because the maneuver failed or because the wall was not perpendicular), because pronation corrected the robot's angle to the wall. Pronation also tried to achieve an optimal berth from the wall; convergence toward the wall would be initiated if the robot was detected diverging from the wall, or if the robot was too far from the wall. `Optimum' distance was defined as a band of ranges that were `far enough to prevent collision with left wall', but `close enough that the followed wall was not mistaken for `distance\_max''.

Behaviour inhibition was implemented by modeling maneuver primitives as queues of motor outputs, to be popped on each timestep. Sensors ran every timestep also. Thus if a bump was detected during a maneuver, the motor queue could be cleared, aborting the maneuver in favor of a newer, more appropriate action. Layering was achieved by giving each sensor a priority; for example, bumping required urgent recovery, so could interrupt any other behaviour. The overarching `wall-following' behaviour was an implicit, emergent layer that was achieved by subsuming all lower behaviours whenever sensors deemed appropriate.

Overall the robot was optimized for reaction at speed; its architecture allowed it to interrupt maneuvers. Its full complement of motors allowed it to make fine steering adjustments without compromising speed. Its static sensor mounting meant that sensing was immediate, and not a bottleneck on choosing action. Its pronation behaviour enabled it to maintain a predictable position in a changing environment, and one designed to reduce collisions so that it could spend less time recovering from them.

We evaluated part of our robot's reaction effectiveness, by measuring its pronation effectiveness. Since the goal was to stay parallel to the wall, we measured how quickly it could achieve a parallel angle after rounding a corner, and how parallel that angle was. This angle change was measured by placing an array of rulers along a wall, reading what distance the robot was from the wall as it passed each one, and using trigonometry between those values to measure angles. As an independent variable, we varied the angle around which the robot had to turn to meet the new wall (ie, how obtuse the two walls were).

Use p values?

\section{Results}

The results section describes the outcomes.  This should be purely
factual descriptions, including qualitative outcomes, quantitative
outcomes and possibly statistics.  For example, you could report the
average speed around a circuit in two conditions plus standard
deviations and a significance test to tell whether you have evidence
that the conditions lead to different results. {\em For coursework~1, this
must include video.}  Typically, the results section can be
surprisingly short, since the Approach section is the one giving
details, this is purely and only factual outcomes.

With respect to your own results, if you describe a reasonably-well
working system in a comprehensible manner you will pass.  If you
competently fill in all of these sections as described in this
specification, you will get at least~55.  Getting a mark over~70
requires demonstrating insight, creativity and / or understanding that
goes beyond the basics laid out for you in this document.  For
example, an insightful comment about one or more cited papers
supported by evidence from your experience might get you these extra
marks.  So might a particularly accurate and replicable account of
your approach and results.

\section{Discussion}

The discussion is the most discursive part of your paper, it {\em may}
include speculation. You should discuss the extent to which your
results addressed the questions described in your introduction, and
what the results imply about your own work and work more broadly.  You
might suggest other experimental protocols that could have given
different results and lessons learned.  This can be a longer section
as well.

\section{Conclusion}
The conclusion is just one paragraph.  After possible digressions in
the discussion, you should come back to state exactly what you tried
to do (brief summary of the introduction), what the outcome was (brief
summary of the results), and what you can certainly state as a
result of this (the implications of the results in light of the introduction.)

%\bibliographystyle{apalike}
%\bibliography{biblio}
\printbibliography

\end{document}
